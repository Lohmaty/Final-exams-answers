Final-exams-answers
===================

\documentclass[12pt]{article}
% Эта строка — комментарий, она не будет показана в выходном файле
\usepackage{ucs}
\usepackage[utf8x]{inputenc} % Включаем поддержку UTF8
\usepackage[russian]{babel}  % Включаем пакет для поддержки русского языка
\title{\LaTeX}
\date{}
\author{}
 
\begin{document}
  \maketitle
  \LaTeX{} "--- это своего рода препроцессор текста для \TeX{} "---
  программы компьютерной вёрстки. \LaTeX{} является программируемым и
  расширяемым, что позволяет автоматизировать большую часть аспектов
  набора, включая нумерацию, перекрёстные ссылки, таблицы и изображения
  (их размещение и подписи к ним), общий вид страницы, библиографию и
  многое-многое другое. \LaTeX{} был первоначально написан Лэсли Лампортом
  в 1984-м году и стал наиболее популярным способом использования \TeX{}а;
  очень мало людей сегодня пишут на оригинальном \TeX{}е. Текущей
  версией является \LaTeXe.
  \newline
  \begin{eqnarray}
    Eпокоя &=& mc^2\\
%Энергия покоя тела равняется произведению массы данного тела на квадрат скорости света. Масса тела является величиной постоянной и не зависит от выбора системы отсчета. Понятие "масса покоя" неприемлемо.
 
  \end{eqnarray}
\end{document}